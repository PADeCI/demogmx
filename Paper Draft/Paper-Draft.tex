\documentclass[
]{jss}

\usepackage[utf8]{inputenc}

\author{
David U. Garibay-Treviño\\Center for Research and\\
Teaching in Economics (CIDE) \And Fernando Alarid-Escudero\\Center for
Research and\\
Teaching in Economics (CIDE)
}
\title{A mexican demographic information Package \pkg{demogmx}}

\Plainauthor{David U. Garibay-Treviño, Fernando Alarid-Escudero}
\Plaintitle{A mexican demographic information Package demogmx}
\Shorttitle{\pkg{demogmx}: A Mexican demographic Package}


\Abstract{
With information from CONAPO and INEGI we made a new package that brings
data about births, migration, population, deaths and aging rate for each
mexican state as well as for the entire country.
}

\Keywords{keywords, not capitalized, \proglang{Java}}
\Plainkeywords{keywords, not capitalized, Java}

%% publication information
%% \Volume{50}
%% \Issue{9}
%% \Month{June}
%% \Year{2012}
%% \Submitdate{}
%% \Acceptdate{2012-06-04}

\Address{
    David U. Garibay-Treviño\\
    Center for Research and Teaching in Economics\\
    First line\\
Second line\\
  E-mail: \email{david.garibay@cide.edu}\\
  URL: \url{http://rstudio.com}\\~\\
    }


% tightlist command for lists without linebreak
\providecommand{\tightlist}{%
  \setlength{\itemsep}{0pt}\setlength{\parskip}{0pt}}




\usepackage{amsmath}

\begin{document}



\hypertarget{introduction}{%
\section{Introduction}\label{introduction}}

This template demonstrates some of the basic LaTeX that you need to know
to create a JSS article.

\hypertarget{code-formatting}{%
\subsection{Code formatting}\label{code-formatting}}

In general, don't use Markdown, but use the more precise LaTeX commands
instead:

\begin{itemize}
\item
  \proglang{Java}
\item
  \pkg{plyr}
\end{itemize}

One exception is inline code, which can be written inside a pair of
backticks (i.e., using the Markdown syntax).

If you want to use LaTeX commands in headers, you need to provide a
\texttt{short-title} attribute. You can also provide a custom identifier
if necessary. See the header of Section \ref{r-code} for example.

\section[R code]{\proglang{R} code}\label{r-code}

Can be inserted in regular R markdown blocks.

\begin{CodeChunk}
\begin{CodeInput}
R> x <- 1:10
R> x
\end{CodeInput}
\begin{CodeOutput}
 [1]  1  2  3  4  5  6  7  8  9 10
\end{CodeOutput}
\end{CodeChunk}

\subsection[Features specific to rticles]{Features specific to
\pkg{rticles}}\label{features-specific-to}

\begin{itemize}
\tightlist
\item
  Adding short titles to section headers is a feature specific to
  \pkg{rticles} (implemented via a Pandoc Lua filter). This feature is
  currently not supported by Pandoc and we will update this template if
  \href{https://github.com/jgm/pandoc/issues/4409}{it is officially
  supported in the future}.
\item
  Using the \texttt{\textbackslash{}AND} syntax in the \texttt{author}
  field to add authors on a new line. This is a specific to the
  \texttt{rticles::jss\_article} format.
\end{itemize}

\begin{small}
\begin{equation}\label{eq:si_hp_ode}
\begin{split}
\frac{dP_{1, j}}{dt} & = b_{j}(t) - \left(d_{1, j}(t)+ \eta_{1, j}(t) + \theta_{1, j}(t) + \mu_{1, j}(t) + \mu_{1, j}^{H}(t)\right) P_{1, j},  \hspace{0.2cm} j = male, female\\
\frac{dP_{i, j}}{dt} & = d_{i-1, j}(t)P_{i-1, j} - \left(d_{i, j}(t) + \eta_{i, j}(t) +         \theta_{i, j}(t) + \mu_{i, j}(t) + \mu_{i, j}^{H}(t)\right) P_{i, j}, \hspace{0.2cm} i=2,\ldots,n \hspace{0.1cm} ; \hspace{0.1cm} j = male, female \\
\frac{dDOC_{i, j}}{dt} & = \mu_{i, j}(t) P_{i, j}, \hspace{0.2cm} i = 1,\ldots, n \hspace{0.2cm} and \hspace{0.2cm} j = male, female \\ %\text{ for all } i = 1, \ldots, n,\\
\frac{dDH_{i, j}}{dt} & = \mu^{H}_{i, j}(t) P_{i, j}, \hspace{0.2cm} i = 1, \ldots, n  \hspace{0.2cm} and \hspace{0.2cm} j = male, female %\text{ for all } i = 1, \ldots, n,\\
\end{split}
\end{equation}
\end{small}

\begin{small}
\begin{equation}\label{eq:si_hp_ode}
\begin{split}
\frac{dP_{1, j}}{dt} & = b_{j}(t) - \left(d_{1, j}(t)+ \eta_{1, j}(t) + \theta_{1, j}(t) + \mu_{1, j}(t) + \mu_{1, j}^{H}(t)\right) \\
\frac{dP_{i, j}}{dt} & = d_{i-1, j}(t)P_{i-1, j} - \left(d_{i, j}(t) + \eta_{i, j}(t) +         \theta_{i, j}(t) + \mu_{i, j}(t) + \mu_{i, j}^{H}(t)\right) \\
\frac{dDOC_{i, j}}{dt} & = \mu_{i, j}(t) \\
\frac{dDH_{i, j}}{dt} & = \mu^{H}_{i, j}(t) \\
\end{split}
\end{equation}
\end{small}

where \(P_{i, j}\) is the population in age group \(i\) and sex group
\(j\), where \(DOC_{i, j}\) is the number of deaths from other causes in
age group \(i\) and sex group \(j\), \(DH_{i, j}\) is the number of
homicides in age group \(i\) and sex group \(j\), \(d_{i, j}(t)\) is the
rate of aging from age group \(i\) to age group \(i+1\),
\(\mu_{i, j}(t)\) is the background mortality for age group \(i\) and
sex group \(j\) in year \(t\) and \(\mu_{i, j}^{H}(t)\) is the homicide
rate for age group \(i\) and sex group \(j\) in year \(t\),
\(\eta_{i, j}\) represents migration rate for age group \(i\) and sex
group \(j\), while \(\theta_{i, j}\) is the immigration rate for age
group \(i\) and sex group \(j\).




\end{document}
